\documentclass[UTF8, 12pt, a4paper]{article}
\usepackage[T1]{fontenc}
	\usepackage{ctex}
	\author{王东岳}
	\title{ST!文档}
\begin{document}
	\maketitle
	\newpage
	\section{内存模型}
		\subsection{基本数据结构}
			\paragraph{}
				基本数据结构包括obect和builtinFunc.
		\subsubsection{object}
			\paragraph{}
				object是所有对象的基础,在ST!语言中,object拥有以下属性:
				\\ size
				\\ dict
		\subsubsection{builtinFunc}
			\paragraph{}
				builtinFunc是所有内建函数的统一表示,继承自object,在其上添加了属性:
				\\ exec
		\subsection{内建数据类型}
			\paragraph{}
				内建数据类型是用其他语言编写的用于实现扩展功能的数据类型,
				通过对object的继承来实现各种功能,
				同时通过builtinFunc来定义方法
		\subsection{垃圾收集机制}
			\paragraph{}
				垃圾收集机制是固化在object的dict中的,使用引用计数垃圾收集来完成任务
		\subsection{object的基础方法}
			\paragraph{}
				object的dict内存有一个\_\_state\_\_存储其状态(默认为object),
				object需要的dict存储其各类基础函数
			\\ \_\_getAttribute\_\_
			\\ \_\_setAttribute\_\_
			\\ \_\_toStr\_\_
			\\ \_\_hash\_\_
			\\ \_\_equal\_\_
			\\ \_\_namespace\_\_
	\section{基本内建类型}
		\subsection{基本内建类型介绍}
			\paragraph{}
				基本内建类型包括int, double, bool,
				builtinFunc, progFunc, classFunc, state, namespace
		\subsection{int, double}
			\paragraph{}
				整数类型与浮点数类型,实现以下操作:
				\\parse
				\\ + - * / **
				\\ == != > < >= <=
				对于整数,还有一个\%
				具体意义与python相同
		\subsection{string}
			\paragraph{}
				字符串类型,实现以下操作:
				\\substr
				\\ + * =
				\\ == != > < >= <=
		\subsection{bool}
			\paragraph{}
				是非类型,用于判断,实现以下操作:
				\\ \&\& || ! \^
				\\ = == !=
		\subsection{builtinFunc, progFunc, classFunc}
			\paragraph{}
				继承自func,实现其exec,用于各种地方的函数表示
		\subsection{state, namespace}
			\paragraph{}
				重载其中的\_\_getAttribute\_\_和\_\_setAttribute\_\_
		\section{运行模型}
			\paragraph{}
				程序:语句*
				\\语句:简单语句|语句块
				\\简单语句:for/while/if/skip/表达式
				\\语句块:{语句*}




\end{document}
